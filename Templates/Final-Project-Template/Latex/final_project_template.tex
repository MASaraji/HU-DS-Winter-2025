\documentclass[12pt]{article}

% --- Packages ---
\usepackage[utf8]{inputenc}
\usepackage[T1]{fontenc}
\usepackage{lmodern}
\usepackage{geometry}
\usepackage{graphicx}
\usepackage{hyperref}
\usepackage{xcolor}
\usepackage{amsmath,amssymb}
\usepackage{listings}
\usepackage{enumitem}
\usepackage{titlesec}
\usepackage{fancyhdr}
\usepackage{fontawesome}
\usepackage{tikz}
\usepackage{lipsum}
\usetikzlibrary{shapes.geometric, arrows}
\usepackage{xepersian}
% --- Persian Setup ---
\settextfont{Vazirmatn-Light}
\setlatintextfont{Times New Roman}
\reversemarginpar

% --- Color Palette ---
\definecolor{primary}{RGB}{167,146,119}
\definecolor{secondary}{RGB}{209,187,158}
\definecolor{accent}{RGB}{234,216,192}

% --- Page Geometry ---
\geometry{a4paper, margin=1in, headheight=15pt}

% --- Title Formatting ---
\titleformat{\section}[block]
{\color{primary}\normalfont\LARGE\bfseries}
{\thesection}{1em}{}

\titleformat{\subsection}[block]
{\color{secondary}\normalfont\Large\bfseries}
{\thesubsection}{1em}{}

% --- Header/Footer ---
\pagestyle{fancy}
\fancyhf{}
\renewcommand{\headrulewidth}{0.5pt}
\fancyhead[R]{\textcolor{primary}{\nouppercase{\leftmark}}}
\fancyhead[L]{\includegraphics[height=0.8cm]{university-logo}}
\fancyfoot[C]{\textcolor{primary}{\thepage}}

% --- Code Listings ---
\lstset{
	language=Python,
	basicstyle=\ttfamily\footnotesize,
	numbers=left,
	numberstyle=\tiny\color{gray},
	stepnumber=1,
	numbersep=5pt,
	backgroundcolor=\color{white},
	showspaces=false,
	showstringspaces=false,
	frame=single,
	rulecolor=\color{lightgray},
	breaklines=true,
	tabsize=2,
	captionpos=b,
	keywordstyle=\color{secondary},
	commentstyle=\color{gray},
	stringstyle=\color{accent}
}


% --- Title Page ---
\title{
	\vspace*{-2cm}
	\includegraphics[width=0.2\textwidth]{university-logo}\\
	\vspace{1cm}
	\textcolor{primary}{\bfseries  گزارش پروژه پایانی}\\
	\Huge ساختمان داده و الگوریتم
}
\author{
	\large
	\textbf{نام و نام خانوادگی:} \hrulefill \\
	\vspace{0.5cm}
	\textbf{شماره دانشجویی:} \hrulefill \\
	\vspace{1cm}
	\textcolor{primary}{\faCalendar\ تاریخ تحویل: \today}
}
\date{}

\begin{document}
	
	\maketitle
	\thispagestyle{empty}
	
	% --- Table of Contents ---
	\newpage
	\renewcommand{\contentsname}{\textcolor{primary}{فهرست مطالب}}
	\tableofcontents
	\newpage
	
	% ========================
	% Main Content
	% ========================
	
	\section{ساختار نرم افزار}
	\subsection{ساختار کلی نرم افزار}
	توضیحات: [نرم افزار شما چه بخش هایی دارد. تا چه بخشی از پروژه انجام شده است. هر فایل مربوط به چه بخشی می شود.]
	\subsection{کلاس ها}
	توضیحات: [خلاصه کلاس های اصلی نرم افزار و کارکرد های آن ها]
	\begin{itemize}[noitemsep]
		\item \textbf{کلاس شماره ۱: [نام]}\\
		\lipsum[8]
		\item \textbf{کلاس شماره ۲: [نام]}\\
		\lipsum[8]
		
	\end{itemize}
	\newpage
	\section{ساختمان های داده و الگوریتم ها}
\subsection{ساختمان های داده}
\begin{itemize}[noitemsep]
	\item \textbf{ساختمان داده شماره ۱: [نام]}\\
	توضیحات: [چرا این ساختمان داده مناسب است, مزیت ها و عملیات ها]\\
	\lipsum[8]
	\\
	\item \textbf{ساختمان داده شماره ۲: [نام]}\\
	توضیحات: [چرا این ساختمان داده مناسب است, مزیت ها و عملیات ها]\\
	\lipsum[8]
\end{itemize}

\subsection{الگوریتم ها}
% Describe the algorithms implemented.
\begin{itemize}[noitemsep]
	\item \textbf{الگوریتم شماره ۱}\\
	توضیحات: [الگوریتم را توضیح دهید و ورودی و خروجی را مشخص کنید]\\
	\lipsum[9]
	\\[0.2cm]
	سودو کد:
	در صورت پیچیده بودن پیاده سازی سودو کد الگوریتم را نیز بنویسید\\[0.2cm]
	پیچیدگی:
	\begin{itemize}[noitemsep]
		\item \textit{پیچیدگی زمانی:} O(...)
		\item \textit{پیچیدگی مکانی:} O(...)
	\end{itemize}		
	\item \textbf{الگوریتم شماره ۲}\\
	توضیحات: [الگوریتم را توضیح دهید و ورودی و خروجی را مشخص کنید]\\
	\lipsum[9]
	\\[0.2cm]
	سودو کد:
	در صورت پیچیده بودن پیاده سازی سودو کد الگوریتم را نیز بنویسید\\[0.2cm]
	پیچیدگی:
	\begin{itemize}[noitemsep]
		\item \textit{پیچیدگی زمانی:} O(...)
		\item \textit{پیچیدگی مکانی:} O(...)
	\end{itemize}
\end{itemize}

\newpage

\section{چالش ها}
\subsection{چالش هایی که در این پروژه با آن مواجه شدید؟}
\lipsum[2]
\subsection{چالش هایی که در پیاده سازی با آن مواجه شدید؟}
\lipsum[1]

\newpage

\section{نظر شما در مورد پروژه چیست؟}
\lipsum[3]
\newpage

\section{منابع}

\lipsum[4]


	
\end{document}