\documentclass[12pt]{article}

% --- Packages ---
\usepackage[utf8]{inputenc}
\usepackage[T1]{fontenc}
\usepackage{lmodern}
\usepackage{geometry}
\usepackage{graphicx}
\usepackage{hyperref}
\usepackage{xcolor}
\usepackage{amsmath,amssymb}
\usepackage{listings}
\usepackage{enumitem}
\usepackage{titlesec}
\usepackage{fancyhdr}
\usepackage{fontawesome}
\usepackage{tikz}
\usepackage{lipsum}
\usetikzlibrary{shapes.geometric, arrows}
\usepackage{xepersian}
% --- Persian Setup ---
\settextfont{Vazirmatn-Light}
\setlatintextfont{Times New Roman}
\reversemarginpar

% --- Color Palette ---
\definecolor{primary}{RGB}{167,146,119}
\definecolor{secondary}{RGB}{209,187,158}
\definecolor{accent}{RGB}{234,216,192}

% --- Page Geometry ---
\geometry{a4paper, margin=1in, headheight=15pt}

% --- Title Formatting ---
\titleformat{\section}[block]
{\color{primary}\normalfont\LARGE\bfseries}
{\thesection}{1em}{}

\titleformat{\subsection}[block]
{\color{secondary}\normalfont\Large\bfseries}
{\thesubsection}{1em}{}

% --- Header/Footer ---
\pagestyle{fancy}
\fancyhf{}
\renewcommand{\headrulewidth}{0.5pt}
\fancyhead[R]{\textcolor{primary}{\nouppercase{\leftmark}}}
\fancyhead[L]{\includegraphics[height=0.8cm]{university-logo}}
\fancyfoot[C]{\textcolor{primary}{\thepage}}

% --- Code Listings ---
\lstset{
	language=Python,
	basicstyle=\ttfamily\footnotesize,
	numbers=left,
	numberstyle=\tiny\color{gray},
	stepnumber=1,
	numbersep=5pt,
	backgroundcolor=\color{white},
	showspaces=false,
	showstringspaces=false,
	frame=single,
	rulecolor=\color{lightgray},
	breaklines=true,
	tabsize=2,
	captionpos=b,
	keywordstyle=\color{secondary},
	commentstyle=\color{gray},
	stringstyle=\color{accent}
}


% --- Title Page ---
\title{
	\vspace*{-2cm}
	\includegraphics[width=0.2\textwidth]{university-logo}\\
	\vspace{1cm}
	\textcolor{primary}{\bfseries  گزارش پروژه پایانی}\\
	\Huge ساختمان داده و الگوریتم
}
\author{
	\large
	\textbf{نام و نام خانوادگی:} \hrulefill \\
	\vspace{0.5cm}
	\textbf{شماره دانشجویی:} \hrulefill \\
	\vspace{1cm}
	\textcolor{primary}{\faCalendar\ تاریخ تحویل: \today}
}
\date{}

\begin{document}
	
	\maketitle
	\thispagestyle{empty}
	
	% --- Table of Contents ---
	\newpage
	\renewcommand{\contentsname}{\textcolor{primary}{فهرست مطالب}}
	\tableofcontents
	\newpage
	
	% ========================
	% Main Content
	% ========================
	
	\section{ساختار نرم افزار}
	\subsection{ساختار فایل‌های نرم افزار}
	توضیحات: [در این بخش، ساختار کلی پروژه خود را به‌صورت یک نمای درختی (Tree Structure) ارائه دهید. این ساختار باید شامل تمامی فولدرها، کلاس‌ها، فایل‌های داده‌ای و سایر فایل‌های مرتبط باشد. لطفاً برای هر فایل، نوع (Type) و نام کامل آن را نیز مشخص کنید. در صورت امکان، تصویری از این ساختار به همراه مسیرها (Paths) و چیدمان فایل‌ها نیز ضمیمه نمایید تا نمای کلی پروژه به‌صورت بصری قابل مشاهده باشد.  ]
	\subsection{کلاس ها}
	توضیحات: [
در این بخش، نمودار کلاس (Class Diagram) نرم‌افزار خود را ارائه دهید. این نمودار باید شامل کلیهٔ کلاس‌های تعریف‌شده در سیستم، روابط بین آن‌ها (مانند ارث‌بری، وابستگی یا پیاده‌سازی رابط‌ها)، متدها و ویژگی‌های اصلی هر کلاس باشد. لطفاً ارتباطات منطقی بین کلاس‌ها را به‌گونه‌ای نمایش دهید که معماری کلی سیستم به‌وضوح قابل درک باشد.

** با توجه به اینکه ممکن است بسیاری از دانشجویان با نمودار کلاس آشنا نباشند می‌توانید به جای آن، فهرستی کامل از تمامی کلاس‌های تعریف‌شده در پروژه خود را به‌همراه نام آن‌ها ارائه دهید. برای هر کلاس، یک توضیح کوتاه درباره نقش آن در نرم‌افزار، متدهای اصلی و ویژگی‌های مهم ارائه شود. ]
	\begin{itemize}[noitemsep]
		\item \textbf{کلاس شماره ۱: [نام]}\\
		\lipsum[8]
		\item \textbf{کلاس شماره ۲: [نام]}\\
		\lipsum[8]
		
	\end{itemize}
	\newpage
	\section{ساختمان های داده و الگویتم‌ها}
\subsection{ساختمان های داده}
توضیحات: [ فهرستی از کلیهٔ ساختمان داده‌هایی که در پروژهٔ خود پیاده‌سازی کرده‌اید ارائه دهید. برای هر ساختمان داده، مسیر دقیق فایل یا پوشه‌ای که پیاده‌سازی در آن انجام شده نیز مشخص شود (برای مثال: ds/mybst برای پیاده‌سازی درخت جست‌وجوی دودویی).
در صورتی که در پروژهٔ خود از ساختمان داده‌ای ابتکاری یا خارج از مباحث مطرح‌شده در طول ترم استفاده کرده‌اید، لازم است در این بخش توضیحات کامل شامل ساختار داده، توابع اصلی و نحوهٔ عملکرد آن را ارائه دهید و مسیر فایل مربوطه را نیز ذکر کنید.
برای ساختمان داده‌هایی که در طول ترم مورد مطالعه قرار گرفته‌اند (مانند لیست پیوندی، صف، پشته، BST و ...) نیازی به شرح ساختاری نیست، تنها ذکر نام و آدرس آن‌ها کفایت می‌کند. ]

\begin{itemize}[noitemsep]
	\item \textbf{ساختمان داده شماره ۱: [نام]}\\
	توضیحات: [چرا این ساختمان داده مناسب است, مزیت ها و عملیات ها]\\
	\lipsum[8]
	\\
	\item \textbf{ساختمان داده شماره ۲: [نام]}\\
	توضیحات: [چرا این ساختمان داده مناسب است, مزیت ها و عملیات ها]\\
	\lipsum[8]
\end{itemize}

\subsection{الگوریتم ها}
% Describe the algorithms implemented.
	توضیحات: [در این بخش، لازم است کلیه الگوریتم‌های کلی (مانند الگوریتم‌های مرتب‌سازی) که در پروژه مورد استفاده قرار گرفته اند یا  الگوریتم‌های ابتکاری خود شما فهرست شوند. برای هر الگوریتم، نام آن و آدرس فایل یا مسیر پوشه‌ای که پیاده‌سازی در آن انجام شده به‌صورت دقیق ذکر گردد. در صورت استفاده از الگوریتم‌هایی با ساختار خاص یا طراحی ابتکاری، توضیحی مختصر درباره نحوه عملکرد آن‌ها و شبه کد نیز ارائه شود. اگر پروژه فاقد هرگونه الگوریتم در این دسته‌ها است، این بخش می‌تواند خالی باقی بماند.]
\begin{itemize}[noitemsep]
	\item \textbf{الگوریتم شماره ۱}\\
	توضیحات: [الگوریتم را توضیح دهید و ورودی و خروجی را مشخص کنید]\\
	\lipsum[9]
	\\[0.2cm]
	سودو کد:
	در صورت پیچیده بودن پیاده سازی سودو کد الگوریتم را نیز بنویسید\\[0.2cm]
	پیچیدگی:
	\begin{itemize}[noitemsep]
		\item \textit{پیچیدگی زمانی:} O(...)
		\item \textit{پیچیدگی مکانی:} O(...)
	\end{itemize}		
	\item \textbf{الگوریتم شماره ۲}\\
	توضیحات: [الگوریتم را توضیح دهید و ورودی و خروجی را مشخص کنید]\\
	\lipsum[9]
	\\[0.2cm]
	سودو کد:
	در صورت پیچیده بودن پیاده سازی سودو کد الگوریتم را نیز بنویسید\\[0.2cm]
	پیچیدگی:
	\begin{itemize}[noitemsep]
		\item \textit{پیچیدگی زمانی:} O(...)
		\item \textit{پیچیدگی مکانی:} O(...)
	\end{itemize}
\end{itemize}

\newpage
\section{ پیاده‌یازی نرم‌افزار}
\subsection{ساختمان داده‌های اصلی }
	توضیحات: [در این بخش، لازم است کلیهٔ ساختمان داده‌هایی که در برنامه تعریف شده‌اند و به‌صورت مستقیم مسئول نگهداری داده‌ها هستند، فهرست گردند. منظور از ساختمان داده‌های این بخش، آن دسته از ساختارهایی‌ست که در طول اجرای نرم‌افزار داده‌هایی نظیر اطلاعات کاربران، لیست سفارشات، پیام‌ها، تراکنش‌ها یا سایر داده‌های اصلی سیستم را در خود ذخیره و مدیریت می‌کنند.
برای هر ساختمان داده، اطلاعات زیر باید به‌صورت دقیق ارائه شود:
نام ساختمان داده
نوع داده‌ای که نگهداری می‌کند (مثلاً: کاربر، پیام، محصول، لاگ سیستم و ...)
مسیر که تعریف شده است (برای مثال: data/structure/UserList.py
line : 17)]\\

	\begin{itemize}[noitemsep]
		\item \textbf{ساختمان داده شماره ۱: [نام]}\\
		\lipsum[8]
		\item \textbf{ساختمان داده شماره ۲: [نام]}\\
		\lipsum[8]
		
	\end{itemize}

 
	\newpage
	
\subsection{عملکرد‌ها}
توضیحات: [در این بخش، لازم است تمامی عملکردهایی که در فایل تعریف پروژه مشخص شده‌اند، به‌صورت فهرست‌شده ارائه گردند. برای هر عملکرد، یکی از وضعیت‌های زیر را مشخص کنید:
اگر عملکرد انجام نشده، عبارت «انجام نشده» قید شود.
اگر عملکرد درست کار نمی‌کند، عبارت «کار نمی‌کند» درج گردد و در ادامه، در یک خط یا کمتر، راه‌حل یا بخشی از مسیر طی‌شده برای پیاده‌سازی آن را شرح دهید.
اگر عملکرد به‌درستی پیاده‌سازی شده، توضیح دهید از طریق کدام تابع (یا متد) و در کدام فایل پیاده‌سازی صورت گرفته است. در ادامه نیز، در یک خط نهایتاً خلاصه‌ای از منطق الگوریتم مورد استفاده را بیان کنید.
از ارائه توضیح علمی یا کلی‌گویی‌هایی مانند «تلاش کردم ولی نشد» خودداری شود. هدف از این بخش، نمایش شفاف میزان تحقق عملکردهای خواسته‌شده به‌صورت دقیق، مختصر و قابل ارزیابی است.
. در نهایت هزینه‌ی زمانی عملکرد های خود را نیز بنویسید.]\\

\begin{itemize}[noitemsep]
	\item \textbf{عملکرد شماره ۱}\\
	مسیرتحقق عملکرد: []
\\

 	منطق و الگوریتم:
\\
	\lipsum[9]
	\\[0.2cm]
	در صورت پیچیده بودن پیاده سازی سودو کد الگوریتم را نیز بنویسید\\[0.2cm]
	پیچیدگی:
	\begin{itemize}[noitemsep]
		\item \textit{پیچیدگی زمانی:} O(...)
		\item \textit{پیچیدگی مکانی:} O(...)
	\end{itemize}		
		\item \textbf{عملکرد شماره ۲}\\
	مسیرتحقق عملکرد: []
\\

 	منطق و الگوریتم:
\\
	\lipsum[9]
	\\[0.2cm]
	در صورت پیچیده بودن پیاده سازی سودو کد الگوریتم را نیز بنویسید\\[0.2cm]
	پیچیدگی:
	\begin{itemize}[noitemsep]
		\item \textit{پیچیدگی زمانی:} O(...)
		\item \textit{پیچیدگی مکانی:} O(...)
	\end{itemize}	
\end{itemize}
	




\newpage
\section{چالش ها}
\subsection{چالش هایی که در این پروژه با آن مواجه شدید؟}
\lipsum[2]
\subsection{چالش هایی که در پیاده سازی با آن مواجه شدید؟}
\lipsum[1]

\newpage

\section{نظر شما در مورد پروژه چیست؟}
\lipsum[3]
\newpage

\section{منابع}

\lipsum[4]


	
\end{document}
